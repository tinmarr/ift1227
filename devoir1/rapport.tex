\documentclass{article}
\usepackage[margin=0.5in]{geometry}
\usepackage{parskip}
\usepackage{amssymb}
\usepackage{graphicx}


\title{Devoir 1}
\author{Martin Chaperot: 20205638\\ Hamza Ali Ousalah: 20249230}
\date{}

\begin{document}

\maketitle

\section*{(a) Table de vérité}
\begin{tabular}{r|c|c|c|c||c|}
    \cline{2-6}
       & A & B & C & D & V \\
    \cline{2-6}
    0  & 0 & 0 & 0 & 0 & 0 \\
    1  & 0 & 0 & 0 & 1 & 0 \\
    2  & 0 & 0 & 1 & 0 & 0 \\
    3  & 0 & 0 & 1 & 1 & 0 \\
    4  & 0 & 1 & 0 & 0 & 0 \\
    5  & 0 & 1 & 0 & 1 & 0 \\
    6  & 0 & 1 & 1 & 0 & 1 \\
    7  & 0 & 1 & 1 & 1 & 1 \\
    8  & 1 & 0 & 0 & 0 & 0 \\
    9  & 1 & 0 & 0 & 1 & 0 \\
    10 & 1 & 0 & 1 & 0 & 1 \\
    11 & 1 & 0 & 1 & 1 & 1 \\
    12 & 1 & 1 & 0 & 0 & 1 \\
    13 & 1 & 1 & 0 & 1 & 1 \\
    14 & 1 & 1 & 1 & 0 & 1 \\
    15 & 1 & 1 & 1 & 1 & 1 \\
    \cline{2-6}
\end{tabular}

\section*{(b) SOP de la fonction V}
\begin{tabular}{|c|c|c|c|c|}
    \hline
    \multicolumn{5}{|c|}{Initial Setup}     \\
    \hline
    Minterm & A     & B     & C     & D     \\
    \hline
    -----   & ----- & ----- & ----- & ----- \\
    \hline
    6       & 0     & 1     & 1     & 0     \\
    10      & 1     & 0     & 1     & 0     \\
    12      & 1     & 1     & 0     & 0     \\
    \hline
    7       & 0     & 1     & 1     & 1     \\
    11      & 1     & 0     & 1     & 1     \\
    13      & 1     & 1     & 0     & 1     \\
    14      & 1     & 1     & 1     & 0     \\
    \hline
    15      & 1     & 1     & 1     & 1     \\
    \hline
\end{tabular}
\quad
\begin{tabular}{|ccccc|}
    \hline
    \multicolumn{5}{|c|}{First Reduction} \\
    \hline
    Pair    & A & B & C & D               \\
    \hline
    (6,7)   & 0 & 1 & 1 & -               \\
    (6,14)  & - & 1 & 1 & 0               \\
    (10,11) & 1 & 0 & 1 & -               \\
    (10,14) & 1 & - & 1 & 0               \\
    (12,13) & 1 & 1 & 0 & -               \\
    (12,14) & 1 & 1 & - & 0               \\
    \hline
    (7,15)  & - & 1 & 1 & 1               \\
    (11,15) & 1 & - & 1 & 1               \\
    (13,15) & 1 & 1 & - & 1               \\
    (14,15) & 1 & 1 & 1 & -               \\
    \hline
\end{tabular}
\quad
\begin{tabular}{|cccccc|}
    \hline
    \multicolumn{6}{|c|}{Second Reduction} \\
    \hline
    Quad          & A & B & C & D &        \\
    \hline
    (6,7,14,15)   & 1 & - & 1 & - & *      \\
    (10,11,14,15) & - & 1 & 1 & - & *      \\
    (12,13,14,15) & 1 & 1 & - & - & *      \\
    \hline
\end{tabular}

\begin{tabular}{|c|c|c|c|c|c|c|c|c|}
    \hline
                     & \multicolumn{8}{|c|}{Minterms}                                                                                                                            \\
    \cline{2-9}
    Prime Implicants & 0110                           & 0111           & 1010           & 1011           & 1100           & 1101           & 1110             & 1111             \\
    \hline
    1-1-             &                                &                & \(\checkmark\) & \(\checkmark\) &                &                & \( \checkmark \) & \( \checkmark \) \\
    -11-             & \(\checkmark\)                 & \(\checkmark\) &                &                &                &                & \( \checkmark \) & \( \checkmark \) \\
    11--             &                                &                &                &                & \(\checkmark\) & \(\checkmark\) & \( \checkmark \) & \( \checkmark \) \\
    \hline
\end{tabular}

\(V(A,B,C,D) = AC + BC + AB\)

\section*{(c) Implementation}
\subsection*{1}
\includegraphics*[width=\textwidth]{impl1}
\subsection*{2}
\includegraphics*[width=\textwidth]{impl2}
\subsection*{3}
\includegraphics*[width=\textwidth]{impl3}


\end{document}
