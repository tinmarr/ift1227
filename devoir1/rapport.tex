\documentclass{article}
\usepackage[margin=0.5in]{geometry}
\usepackage{parskip}
\usepackage{amssymb}


\title{Devoir 1}
\author{Martin Chaperot: 20205638\\ Hamza Ali Ousalah: }
\date{}

\begin{document}

\maketitle

\section*{(a) Table de vérité}
\begin{tabular}{r|c|c|c|c||c|}
    \cline{2-6}
       & A & B & C & D & V \\
    \cline{2-6}
    0  & 0 & 0 & 0 & 0 & 0 \\
    1  & 0 & 0 & 0 & 1 & 0 \\
    2  & 0 & 0 & 1 & 0 & 0 \\
    3  & 0 & 0 & 1 & 1 & 0 \\
    4  & 0 & 1 & 0 & 0 & 0 \\
    5  & 0 & 1 & 0 & 1 & 0 \\
    6  & 0 & 1 & 1 & 0 & 1 \\
    7  & 0 & 1 & 1 & 1 & 1 \\
    8  & 1 & 0 & 0 & 0 & 0 \\
    9  & 1 & 0 & 0 & 1 & 1 \\
    10 & 1 & 0 & 1 & 0 & 1 \\
    11 & 1 & 0 & 1 & 1 & 1 \\
    12 & 1 & 1 & 0 & 0 & 1 \\
    13 & 1 & 1 & 0 & 1 & 1 \\
    14 & 1 & 1 & 1 & 0 & 1 \\
    15 & 1 & 1 & 1 & 1 & 1 \\
    \cline{2-6}
\end{tabular}

\section*{(b) SOP de la fonction V}
\begin{table}[h!]
    \centering
    \begin{tabular}{|c|c|c|c|c|}
        \hline
        \# de 1 & Minterm                 & Binaire                      & Taille 2             & Taille 4 \\
        \hline
        1       & -----                   & -----                        & -----                & -----    \\
        \hline
        2       & \begin{tabular}{c}
                      m6 \\ m10 \\ m12
                  \end{tabular}      & \begin{tabular}{c}
                                           0110 \\ 1010 \\ 1100
                                       \end{tabular}         & \begin{tabular}{c|c}
                                                                   m(6,7)   & 011- \\
                                                                   m(10,11) & 101- \\
                                                                   m(12,13) & 110-
                                                               \end{tabular} & \begin{tabular}{c|c}
                                                                                   m(10,11,14,15) & 1-1- * \\
                                                                                   m(6,7,14,15)   & -11- * \\
                                                                                   m(12,13,14,15) & 11-- *
                                                                               \end{tabular}     \\
        \hline
        3       & \begin{tabular}{c}
                      m7 \\ m11 \\ m13 \\ m14
                  \end{tabular} & \begin{tabular}{c}
                                      0111 \\ 1011 \\ 1101 \\ 1110
                                  \end{tabular} & \begin{tabular}{c|c}
                                                               &      \\
                                                               &      \\
                                                      m(14,15) & 111- \\
                                                               &      \\
                                                               &      \\
                                                  \end{tabular} & -----                                    \\
        \hline
        4       & m15                     & 1111                         & -----                & -----    \\
        \hline
    \end{tabular}
    \caption{Table de minterms}
\end{table}

\begin{table}[h!]
    \centering
    \begin{tabular}{|c|c|c|c|c|c|c|c|c|}
        \hline
        Implicant      & m6             & m7             & m10            & m11            & m12            & m13            & m14              & m15              \\
        \hline
        m(10,11,14,15) &                &                & \(\checkmark\) & \(\checkmark\) &                &                & \( \checkmark \) & \( \checkmark \) \\
        m(10,11,14,15) & \(\checkmark\) & \(\checkmark\) &                &                &                &                & \( \checkmark \) & \( \checkmark \) \\
        m(10,11,14,15) &                &                &                &                & \(\checkmark\) & \(\checkmark\) & \( \checkmark \) & \( \checkmark \) \\
        \hline
    \end{tabular}
    \caption{Prime Implicants}
\end{table}

\(V(A,B,C,D) = AC + BC + AB\)

\end{document}
